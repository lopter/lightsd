\documentclass{lgtdslides}

\usepackage{tikz}
\usepackage{tikzsymbols}
\usepackage[francais]{babel}

\usepackage{lgtdfigs}

\title{S'amuser avec la lumière}
\subtitle{\textit{Avec des ampoules LIFX et un Monome 128}}
\date{While 42 SF, 2017-05-18}
\author{Louis Opter <louis@opter.org>}

\tikzset{arrow/.style={->, >=stealth,ultra thick,rounded corners}}

\begin{document}

\begin{frame}\titlepage\end{frame}

\section{Intro}

\begin{frame}{\LARGE{\texttt{\$ whoami}}}
Bonjour, moi c'est Louis (Opter) et :\vspace{1em}
\begin{itemize}
\item je bosse sur SF depuis \textasciitilde2012 (docker/Uber) ;
\item j'y connais rien en hardware ;
\item donc j'ai acheté des trucs tous faits « simples » ;
\item et voici une démo de ce que j'ai fait.
\end{itemize}
\end{frame}

\section{Démo}

\begin{frame}{Sous vos yeux}
\begin{center}
\begin{tikzpicture}
\pic (0, 0) {monolightarch};
\end{tikzpicture}
\end{center}
\end{frame}


\begin{frame}{Effet sablier/alerte}
\only<1>{Une autre idée que j'aimerais implémenter\ldots}
\begin{center}
\begin{tikzpicture}[overlay,scale=1.2]
\onslide<2->{%
\pic (0, 0) {monome={scale 1.2}};

\coordinate (NW) at (-4, 1.75);
\coordinate (caption) at ($(NW) + (-1.155,0.65)$);

\foreach \x in {-4,-3.5,...,-1}
\fill[mbutton] (\x, -1.75) rectangle +(0.36, -0.36); % function row

\node[rectangle] (b117) at (-1.32,-1.93) {};
\node[rectangle] (b118) at (-0.82,-1.93) {};
}
\onslide<2,4->{%
% targets toggles:
\fill[mbutton] (-4, 1.75) rectangle +(0.36, -0.36);
\foreach \x in {-2,0,...,3.5}
\fill[mbuttvlow] (\x, 1.75) rectangle +(0.36, -0.36);

% h:
\fill[mbutthigh] (-4, -0.25) rectangle +(0.36, -0.36);
\fill[mbutton] (-4, -0.75) rectangle +(0.36, -0.36);
\fill[mbutton] (-4, -1.25) rectangle +(0.36, -0.36);

% s:
\fill[mbuttmed] (-3.5, -0.75) rectangle +(0.36, -0.36);
\fill[mbutton] (-3.5, -1.25) rectangle +(0.36, -0.36);

% b:
\foreach \y in {1.25,0.75,...,-1.25}
\fill[mbutton] (-3, \y) rectangle +(0.36, -0.36);

% k:
\fill[mbutthigh] (-2.5, -1.25) rectangle +(0.36, -0.36);
}
\onslide<2>{%
\draw (caption) node[right] {Ajoutons deux nouvelles fonctions :};

\node (timer) [below=0.5cm of b117] {timer};
\node (alert) [below=0.9cm of b118] {alert};
\draw[arrow] (timer) -- (b117);
\draw[arrow] (alert) -- (b118);
}
\onslide<3>{%
\draw (caption) node[right] {Sélection du compte à rebours:};
\node[below right] at ($(caption.south) + (0, -0.09)$) {\footnotesize{(1 bouton allumé = 1 unité de temps)}};

% partially lit grid:
\foreach \x in {-4,-3.5,...,-0.5}
\foreach \y in {1.75,1.25,...,-1.25}
\fill[mbutton] (\x, \y) rectangle +(0.36, -0.36);
\foreach \y in {1.75,1.25,...,-0.5}
\fill[mbutton] (0, \y) rectangle +(0.36, -0.36);

\foreach \x in {-4,-3.5,...,3.5}
\fill[mbuttoff] (\x, -1.75) rectangle +(0.36, -0.36); % blank function row

\fill[mbutton] (-1.5, -1.75) rectangle +(0.36, -0.36); % time button
\fill[mbutton] (0, -1.75) rectangle +(0.36, -0.36); % dec time scale
\fill[mbutton] (0.5, -1.75) rectangle +(0.36, -0.36); % inc time scale

%\node[rectangle] (b117) at (-1.32,-1.93) {};
%\node[rectangle] (b118) at (-0.82,-1.93) {};
%\node[rectangle] (b119) at (-0.32,-1.93) {};
\node[rectangle] (b120) at (0.18,-1.93) {};
\node[rectangle] (b121) at (0.68,-1.93) {};
\node[rectangle] (b122) at (1.18,-1.93) {};
\draw[ultra thick,decorate,decoration={name=brace,mirror}]
  ($(b120.south west) + (-0.1,-0.35)$) -- ($(b121.south east) + (0.1,-0.35)$);
\node (timectl) [below=0.5cm of b121] {déc/inc de l'unité de temps};
}
\onslide<4>{%
\draw (caption) node[right] {Sélection de la cible et de la fonction (alert):};

\node (alert) [below=0.5cm of b118] {alert};
\draw[arrow] (alert) -- (b118);

\node[rectangle,opacity=0] (b0) at (-3.82,1.57) {B};
\node[rectangle,opacity=0] (b4) at (-1.82,1.57) {B};
\node[rectangle,opacity=0] (b8) at (0.18,1.57) {B};
\node[rectangle,opacity=0] (b12) at (2.18,1.57) {B};

\coordinate (legend) at ($(caption) + (0.7, -0.15)$);
\coordinate (upturn) at ($(legend) + (0,-1.26)$);
\draw[arrow,<-] (b0.south) -- ++(0,-0.35) -- (upturn) -- (legend.south);
\draw[arrow,<-] (b4.south) -- ++(0,-0.35) -- (upturn) -- (legend.south);
\draw[arrow,<-] (b8.south) -- ++(0,-0.35) -- (upturn) -- (legend.south);
\draw[arrow,<-] (b12.south) -- ++(0,-0.35) -- (upturn) -- (legend.south);
}
\onslide<5>{%
\fill[mbutton] (-3.5, 1.75) rectangle +(0.36, -0.36);

\node[rectangle] (b1) at (-3.32,1.57) {};
\node (feedback) [above right=0.5cm of b1] {Voyant d'activité du sablier};
\draw[arrow] (feedback) -| (b1);
}
\end{tikzpicture}
\end{center}
\end{frame}

\section{À propos du projet}

\begin{frame}{Motivations}
À l'origine du projet :
\vspace{1em}
\begin{itemize}
\item Délais de découverte/utilisation, fiabilité ;
\item Sécurité, vie privée, (dé-)centralisation ;
\item Obsolescence programmée (ou non) ;
\item « Accessibilité ».
\end{itemize}
\end{frame}

\begin{frame}{Résultats}
\only<1,3>{%
Quelques découvertes et confirmations :
\vspace{1em}
\begin{itemize}
\item Les blobs binaires c'est de la merde ;
\item L'IoT s'améliore mais c'est pas encore ça ;
\item Spectre de problèmes vraiment intéressant ;
\item Une méthode d'apprentissage de premier ordre.
\end{itemize}
}
\only<2>{%
\emph{\leavevmode\usebeamertemplate***{itemize item} Spectre de problèmes vraiment intéressant :}
\vspace{1em}
\begin{center}
\setlength{\tabcolsep}{15pt}
\begin{tabular}{ccc}
\textbf{LIFX} & \textbf{lightsd} & \textbf{monolight} \\
\hline
hardware & daemon & GUI \\
\hline
embarqué & C & Python \\
\end{tabular}
\begin{tikzpicture}
\draw[arrow] (0,0) -- node[below,pos=0.6] {Bas-niveau} (-4.8,0);
\draw[arrow] (0,0) -- node[below,pos=0.6] {Haut-niveau} (4.8,0);
\end{tikzpicture}
\end{center}
}
\end{frame}

\section{Merci}

{\setbeamertemplate{headline}{}
\begin{frame}
\begin{center}\Huge{Merci}\end{center}
\vspace{1em}
\begin{center}\Large{\emph{Place aux questions et à la discussion}}\end{center}
\vspace{1em}
\begin{itemize}
\item \Large{\href{https://twitter.com/1opter}{@1opter}}
\item \Large{\emph{\#lightsd} sur IRC (\emph{chat.freenode.net})}
\item \Large{\url{https://www.lightsd.io/}}
\end{itemize}
\end{frame}}

\section{Extras}

\begin{frame}{Table des modèles LIFX}
\begin{tabular}{lll}
\textbf{Génération} & \textbf{Modèle} & \textbf{En vente} \\
\hline
Gen 1 & Original 1000, Color 650 & Non \\
\hline
Gen 2 & Color 1000, White 800 & Oui \\
\hline
Gen 3 & A19, BR30, Z (ruban) & Oui \\
\end{tabular}
\par\vspace{2em}
\begin{tabular}{ll}
\textbf{Génération} & \textbf{Notes} \\
\hline
Gen 1 & Supporte 802.11 et 802.15.4 (inutilisé) \\
\hline
Gen 2 & QCA 4002, AllJoyn, \emph{plante} \\
\hline
Gen 3 & versions + avec IR, \emph{plante toujours} \\
\end{tabular}
\end{frame}

\section{Démo (papier)}

\begin{frame}{La grille}
\begin{center}
\begin{tikzpicture}[overlay]
\pic (0, 0) {monome={scale 1.2}};
\end{tikzpicture}
\end{center}
\end{frame}

\begin{frame}{Ligne de fonctions/scènes}
\begin{center}
\begin{tikzpicture}[overlay,scale=1.2]
\onslide<1->{%
\pic (0, 0) {monome={scale 1.2}};

\foreach \x in {-4,-3.5,...,3.5}
\fill[mbutton] (\x, -1.75) rectangle +(0.36, -0.36);
}
\onslide<2->{%
\foreach \x in {-1.5,-1,...,3}
\fill[mbuttoff] (\x, -1.75) rectangle +(0.36, -0.36);

\fill[color=fgcolor,decoration={name=snake,amplitude=2,segment length=45}]
  decorate {(-4.25,-1.35) -- (4.15,-1.35)} -- (4.15,2) -- (-4.25,2) -- cycle;

\node[rectangle] (b112) at (-3.82,-1.93) {};
\node[rectangle] (b113) at (-3.32,-1.93) {};
\node[rectangle] (b114) at (-2.82,-1.93) {};
\node[rectangle] (b115) at (-2.32,-1.93) {};
\node[rectangle] (b116) at (-1.82,-1.93) {};
\node[rectangle] (b117) at (-1.32,-1.93) {};
\node[rectangle] (b127) at (3.68,-1.93)  {};
}
\onslide<2>{
\node (toggle) [above=1.1cm of b112] {%
\begin{minipage}{2cm}
\begin{center}
on/off

toggle
\end{center}
\end{minipage}};
\node (off) [above=0.7cm of b113] {off};
\node (on) [above=1.1cm of b114] {on};
\node (scenes) [above=0.7cm of b116] {scenes\ldots};
\node (uitoggle) [above=1.1cm of b127] {toggle UI};

\draw[arrow] (toggle) -- (b112);
\draw[arrow] (off) -- (b113);
\draw[arrow] (on) -- (b114);
\draw[arrow] (uitoggle) -- (b127);

\draw[ultra thick,decorate,decoration={name=brace}]
  ($(b115.north west) + (-0.1,0.35)$) -- ($(b116.north east) + (0.1,0.35)$);
}
\onslide<3->{%
\coordinate (b0) at (-4, 1.75);
\draw ($(b0) + (-0.25, 0)$) node[below right] {%
\begin{minipage}{10cm}
D'autres idées :
\vspace{1em}
\begin{itemize}
\item Boutons pour la navigation (pagination\ldots);  % will make sense on the next slide
\item Boutons pour contrôler MPD.
\end{itemize}
\end{minipage}
};
}
\end{tikzpicture}
\end{center}
\end{frame}

\begin{frame}{Panneau de contrôle pour une cible (x4)}
\begin{center}
\begin{tikzpicture}[overlay,scale=1.2]
\onslide<1->{%
\pic (0, 0) {monome={scale 1.2}};

\foreach \x in {-4,-3.5,...,-2.5}
\foreach \y in {1.75,1.25,...,-1.25}
\fill[mbutton] (\x, \y) rectangle +(0.36, -0.36);

\foreach \x in {-2,-1.5,...,-0.5}
\foreach \y in {1.75,1.25,...,-1.25}
\fill[mbuttmed] (\x, \y) rectangle +(0.36, -0.36);

\foreach \x in {0,0.5,...,1.5}
\foreach \y in {1.75,1.25,...,-1.25}
\fill[mbutton] (\x, \y) rectangle +(0.36, -0.36);

\foreach \x in {2,2.5,...,3.5}
\foreach \y in {1.75,1.25,...,-1.25}
\fill[mbuttmed] (\x, \y) rectangle +(0.36, -0.36);

\foreach \x in {-4,-3.5,...,3.5}
\fill[mbuttoff] (\x, -1.75) rectangle +(0.36, -0.36);
}
\onslide<2->{%
\foreach \y in {1.75,1.25,...,-1.25}
\fill[mbuttoff] (-2, \y) rectangle +(0.36, -0.36);

\fill[color=fgcolor,decoration={name=snake,amplitude=2,segment length=45}]
  decorate {(-1.75,2) -- (-1.75,-2.36)} -- (4.2,-2.36) -- (4.2,2) -- cycle;

\foreach \x in {-3.5,-3,...,-2.5}
\fill[mbuttoff] (\x, 1.75) rectangle +(0.36, -0.36);

\foreach \y in {1.25,0.75,...,-0.25} % h
\fill[mbuttoff] (-4, \y) rectangle +(0.36, -0.36);
\fill[mbutthigh] (-4, -0.25) rectangle +(0.36, -0.36);

\foreach \y in {1.25,0.75,...,-0.75} % s
\fill[mbuttoff] (-3.5, \y) rectangle +(0.36, -0.36);
\fill[mbuttmed] (-3.5, -0.75) rectangle +(0.36, -0.36);

\foreach \y in {1.25,0.75,...,-1.25} % b
\fill[mbutton] (-3, \y) rectangle +(0.36, -0.36);

\foreach \y in {1.25,0.75,...,-1.25} % k
\fill[mbuttoff] (-2.5, \y) rectangle +(0.36, -0.36);
\fill[mbutthigh] (-2.5, -1.25) rectangle +(0.36, -0.36);

\node[rectangle] (b16) at (-3.82,1.07) {};
\node (INC) at (-4.82,1.07) {INC};
\draw[arrow] (INC) -- (b16);

\node[rectangle] (b32) at (-3.82,0.57) {};
\node (inc) at (-4.82,0.57) {inc};
\draw[arrow] (inc) -- (b32);

\node[rectangle] (b80) at (-3.82,-0.93) {};
\node (dec) at (-4.82,-0.93) {déc};
\draw[arrow] (dec) -- (b80);

\node[rectangle] (b96) at (-3.82,-1.43) {};
\node (DEC) at (-4.82,-1.43) {DÉC};
\draw[arrow] (DEC) -- (b96);

\node[rectangle] (b4) at (-1.82,1.57) {};
\node[rectangle] (b5) at (-1.32,1.57) {};
\draw[arrow] (b5) -- (b4.west);
\draw (b5) node[right] {Ligne de fonctions/voyants (toggle\ldots)};

\node[rectangle] (b20) at (-1.82,1.07) {};
\node[rectangle] (b21) at (-1.32,1.07) {};
\draw[arrow] (b21) -- (b20.west);
\draw (-1.32,1.32) node[below right] {%
\begin{minipage}{10cm}
4 barres de contrôle (HSBK):
\vspace{1ex}
\begin{itemize}
\item Hue : 0.0--360.0°;
\item Saturation : 0.0--1.0;
\item (B) Luminosité : 0.0--1.0;
\item (K) Température : 2500--9000K.
\end{itemize}
\end{minipage}};
}
\end{tikzpicture}
\end{center}
\end{frame}
\end{document}
